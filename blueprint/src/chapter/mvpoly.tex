<<<<<<< HEAD
\begin{def}
  We define a polynomial over C to be a diagram $F$ in C of shape
  \begin{equation}
  \lean{MvPoly}
  \label{equ:poly}
  \xymatrix{
  I  & B \ar[l]_s \ar[r]^f & A \ar[r]^t & J \, . }
  \end{equation}
  We define $\poly{F}:\slice I \to \slice J$ as the composite
  \[
  \xymatrix{
  \slice{I} \ar[r]^{\pbk{s}} & \slice{B} \ar[r]^{\radj{f}} & \slice{A} \ar[r]^{\ladj{t}} & \slice{J} \, . }
  \]
  We refer to $\poly{F}$ as the polynomial functor associated to $F$, or
  the extension of $F$, and say that $F$ represents $\poly F$.
\end{def}

\begin{def}
\lean{MvPoly.sum}
\uses{structure:MvPoly}
The sum of two polynomials in many variables.
\end{def}

\begin{def}
\lean{MvPoly.prod}
\uses{structure:MvPoly}
The product of two polynomials in many variables.
\end{def}

\begin{para}\label{para:BC-distr}
  We shall make frequent use of the Beck-Chevalley isomorphisms and
  of the distributivity law of dependent sums over dependent
  products~\cite{MoerdijkI:weltc}.  Given a cartesian square
  \[
  \xymatrix {
  \cdot \drpullback \ar[r]^g \ar[d]_u & \cdot \ar[d]^v \\
  \cdot \ar[r]_f & \cdot
  }
  \]
  the Beck-Chevalley isomorphisms are
  \[
    \ladj g \, \pbk u \iso \pbk v \, \ladj f \qquad \text{and} \qquad
    \radj g \, \pbk u \iso \pbk v \, \radj f \,.
    \]

  Given maps $C \stackrel u \longrightarrow B \stackrel f \longrightarrow A$,
  we can construct the diagram
  \begin{equation}\label{distr-diag}
  \xymatrixrowsep{40pt}
  \xymatrixcolsep{27pt}
  \vcenter{\hbox{
  \xymatrix @!=0pt {
  &N \drpullback \ar[rr]^g \ar[ld]_e \ar[dd]^{w=\pbk f(v)}&& M \ar[dd]^{v=\radj f (u)} \\
  C \ar[rd]_u && & \\
  &B \ar[rr]_f && A\,,
  }}}
  \end{equation}
  where $w = \pbk f\, \radj f(u)$ and $e$ is the counit
  of $\pbk{f} \adjoint \radj{f}$.
  For such diagrams the following
  distributive law holds:
  \begin{equation}\label{distr-law}
  \radj f \, \ladj u \iso \ladj v \, \radj g \, \pbk e \,.
  \end{equation}

\begin{para}
\label{para:comp}
\lean{MvPoly.comp}
\uses{structure:MvPoly}
We now define the operation of substitution of polynomials, and show that the
  extension of substitution is composition of polynomial functors, as expected.
  In particular, the composite of two polynomial functors is again polynomial.
  Given polynomials
\[
\xymatrix{
 &  B \ar[r]^f  \ar[dl]_{s} & A \ar[dr]^{t} & \\
 I \ar@{}[rrr]|{F} & & & J } \qquad
\xymatrix{
 & D \ar[r]^g \ar[dl]_{u} & C \ar[dr]^{v}   \\
J \ar@{}[rrr]|{G}& & & K }
\]
we say that $F$ is a polynomial from $I$ to $J$ (and $G$ from
$J$ to $K$), and
we define $G\circ F$, the substitution of $F$ into $G$, to be the polynomial
$I \leftarrow N \to M \to K$ constructed via this diagram:
\begin{equation}
\label{equ:compspan}
\xycenter{
  &  &  & N \ar[dl]_{n} \ar[rr]^{p} \ar@{}[dr] |{(iv)}
&  & D'
\ar[dl]^{\varepsilon} \ar[r]^{q} \ar@{}[ddr] |{(ii)} &
M  \ar[dd]^{w} &  \\
  &  & B' \ar[dl]_{m} \ar[rr]^{r} \ar@{} [dr]|{(iii)}
&  & A'  \ar[dr]^{k} \ar[dl]_{h}
\ar@{} [dd] |{(i)}
&  &  &  \\
   & B \ar[rr]^f \ar[dl]_{s} & & A \ar[dr]_{t} &   & D \ar[dl]^{u}
\ar[r]^{g} & C \ar[dr]^{v} &    \\
I  &    & &   & J &   &   & K   }
\end{equation}
Square $(i)$ is cartesian, and $(ii)$ is a distributivity diagram
like \eqref{distr-diag}: $w$ is obtained
by applying $\radj{g}$ to $k$, and $D'$ is the pullback of $M$ along $g$.
The arrow $\varepsilon: D' \to A'$ is the $k$-component of the counit of
the adjunction $\ladj g \adjoint \pbk g$.
Finally, the squares $(iii)$ and $(iv)$ are
cartesian.
\end{para}

\begin{proposition}\label{thm:subst}
\lean{MvPoly.comp.functor}
\uses{structure:MvPoly, def:MvPoly.comp}
There is a natural isomorphism
  $$
  \poly{G\circ F} \iso \poly{G} \circ \poly{F} .
  $$
\end{proposition}

\begin{proof}
  Referring to Diagram~\eqref{equ:compspan} we have
  the following chain of natural isomorphisms:
\begin{eqnarray*}
\extension{G} \circ \extension{F} & = & \ladj{v} \, \radj{g} \, \pbk{u}  \; \ladj{t}  \, \radj{f}
\, \pbk{s} \\
& \iso & \ladj{v} \, \radj{g}  \, \ladj{k}  \, \pbk{h}  \, \radj{f}  \, \pbk{s}\\
 & \iso &
\ladj{v} \, \ladj{w}  \, \radj{q}  \, \pbk{\varepsilon}  \, \pbk{h}  \, \radj{f} \, \pbk{s}\\
& \iso &
\ladj{v} \, \ladj{w}  \, \radj{q}  \, \radj{p}  \, \pbk{n}  \, \pbk{m} \, \pbk{s}\\
& \iso &
\ladj{(v\, w)} \, \radj{(q \, p)}  \, \pbk{(s\, m\, n)}\, \\
& = & \extension{G \circ F}\,.
\end{eqnarray*}
Here we used the Beck-Chevalley isomorphism for the cartesian square
$(i)$, the distributivity law for $(ii)$, Beck-Chevalley isomorphism
for the cartesian squares $(iii)$ and $(iv)$, and finally pseudo-functoriality
of the pullback functors and their adjoints.
\end{proof}
=======
Let $\catC$ be category with pullbacks and terminal object.

\begin{definition}[multivariable polynomial functor]\label{defn:Polynomial}
  \lean{CategoryTheory.MvPoly} \leanok
  A \textbf{polynomial} in $\catC$ from $I$ to $O$ is a triple $(i,p,o)$ where
  $i$, $p$ and $o$ are morphisms in $\catC$ forming the diagram
  $$\polyspan IEBJipo.$$
  The object $I$ is the object of input variables and the object $O$ is the object of output
  variables. The morphism $p$ encodes the arities/exponents.
  \end{definition}


\begin{definition}[extension of polynomial functors]
  \label{defn:extension}
  \lean{CategoryTheory.MvPoly.functor} \leanok
  The \textbf{extension} of a polynomial $\polyspan IBAJipo$ is the functor
  $\upP = o_! \, f_* \, i^* \colon \catC / I \to \catC/ O$. Internally, we can define $\upP$ by
  $$\upP \seqbn{X_i}{i \in I} = \seqbn{\sum_{a \in A_j} \prod_{b \in B_a} X_{s(b)}}{j \in J}$$
  A \textbf{polynomial functor} is a functor that is naturally isomorphic to the extension of a polynomial.
  \end{definition}
>>>>>>> refs/remotes/origin/master
